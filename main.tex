\documentclass[11pt,a4paper,sans]{moderncv}
\moderncvstyle{banking}  % Stile del CV
\moderncvcolor{burgundy}     % Colore principale coerente con la versione Word
\usepackage[scale=0.85]{geometry} % Margini
\usepackage[dvipsnames]{xcolor}
\usepackage{tcolorbox} % Per sfondi alle sezioni
%\usepackage{multicol}\setlength{\columnsep}{1cm}

% ---- Informazioni personali ---- %
\name{Alessandro}{Sardone}
\homepage{sardonealessandro.com}
%\phone{+39 3926607129}
\email{alessandro.sardone@iwh-halle.de}
\address{Landsberger Straße 59, 06112 Halle (Saale), Germany}
\extrainfo{Citizenship: Italian}

\begin{document}
\makecvtitle

% ---- Summary ---- %
\section{Summary}
\begin{tcolorbox}[colback=gray!20, colframe=gray!50, arc=4mm]
\textbf{Ph.D. candidate} at the Halle Institute for Economic Research (IWH) and Martin-Luther-University of Halle-Wittenberg, contributing to IWH forecast reports on monetary conditions in the Euro area and sectoral $CO_2$ emissions in Germany. Co-organizer of the IWH-CIREQ-GW-BOKERI Macroeconometric Workshop.
\textbf{Research interests}: Environmental macroeconomics, monetary and fiscal policy, and inequality.
Visiting researcher at Georgia State University (2024).
\textbf{On the Job Market 2025/2026}.
\end{tcolorbox}

% ---- Education ---- %
\section{Education}
\cventry{2020 -- Present}{Ph.D. candidate in Economics}{IWH-DPE \& Martin-Luther-University, Halle (DE)}{}{}{
Supervisors: Oliver Holtem\"oller, Roland Winkler\newline
Dissertation: ``Environmental macroeconomics: distributional effects, fiscal policies, and the role of central banks''}

\cventry{Spring 2024}{Research Visit}{Georgia State University (US)}{}{}{Host: Garth Heutel}{}

\cventry{2016 -- 2019}{M.Sc. in Economics -- double degree (with Honors)}{University of Trento (IT) \& Friedrich-Schiller University Jena (DE)}{}{}{}
\cventry{2012 -- 2016}{B.A. in Economics and Management}{University of Trento (IT)}{}{}{}

% ---- Publications ---- %
\section{Job Market Paper}
\cvitem{}{\textbf{Road to Net Zero: Carbon Policy and Redistributional Dynamics in the Green Transition}[\href{https://drive.google.com/file/d/1hKzybkZe7f7aWS2zyRy5PPupZMZY-1M5/view?usp=share_link}{\textcolor{Maroon}{latest version}}]}
Abstract:\\
This paper explores the macroeconomic and distributional effects of the European Union's transition to Net Zero emissions through a gradually increasing carbon tax. The analysis is conducted within an Environmental DSGE model featuring two household types and distinct energy and non-energy sectors. Five alternative uses of carbon tax revenues are considered: equal transfers to households, targeted transfers to Hand-to-mouth households, subsidies to green energy firms, and reductions in labor and capital income taxes. The results indicate that, absent technological progress, the carbon tax induces a recession, reduces investment and consumption, and results in permanently higher energy prices. However, aggregate inflation turns deflationary after the initial adjustment. Different recycling schemes reveal a clear trade-off between efficiency and equity. Targeted transfers are the most progressive, but they entail the largest macroeconomic costs. Subsidies and tax cuts, on the other hand, mitigate output and investment, but they are more regressive. When agents revise expectations at intermediate policy targets, adjustment paths become more volatile, and inflation temporarily rises around announcements. These findings underscore the importance of designing policies that balance efficiency and equity during the Net Zero transition.

\section{Publications \Large{\normalfont{(Peer-reviewed)}}}
\cvitem{2022}{\textbf{Inflation puzzles, the Phillips Curve and output expectations: New perspectives from the Euro Zone} (with G. Passamani, R. Tamborini), \textit{Empirica} [\href{https://doi.org/10.1007/s10663-021-09515-8}{\textcolor{Maroon}{article}}], 2022}

% ---- Working Papers ---- %
\section{Working Papers}
\cvitem{}{\textbf{Monetary Policy Rules and Environmental Fiscal Policy in a Two-Sector DSGE Model} (with O. Holtemöller), 2024 [\href{https://www.econstor.eu/bitstream/10419/301153/1/1897940394.pdf}{\textcolor{Maroon}{working paper}}] [\href{https://drive.google.com/file/d/1b8wpT3_NofSwkxtJ3hm8I0AkgkUq9pNq/view}{\textcolor{Maroon}{latest version}}]}

% ---- Work in Progress ---- %
\section{Work in Progress}
\cvitem{}{\textbf{Environmental Dynamic Stochastic General Equilibrium Models} (with B. Annicchiarico, S. Carattini, C. Fischer, G. Heutel and I. Mourelon)}

% ---- Policy Work ---- %
\section{Selected Policy Work}
\cvitem{2025}{Recovery on shaky ground -- tariffs dampen growth, but a change in fiscal policy is on the way, \textit{IWH Forecast Reports 13(3), 66-102}}
%\cvitem{2025}{Economic recovery in Germany -- but structural problems and US trade policy weigh on the economy, \textit{IWH Forecast Reports 13(2), 35-64}}
\cvitem{2025}{A turning point for the German economy?, \textit{IWH Forecast Reports 13(1), 1-33}}
\cvitem{2024}{Medium-term projections of macroeconomic developments and scenarios for achieving statutory emission targets, \textit{IWH Forecast Reports 12(4), 170-187}}
%\cvitem{2024}{Frosty prospects for the German economy, \textit{IWH Forecast Reports 12(4), 127-169}}
%\cvitem{2024}{Moderate economic growth in the world – German economy continues to stagnate, \textit{IWH Forecast Reports 12(3), 94-123}}
%\cvitem{2024}{German economy still on the defensive – but first signs of an end to the downturn, \textit{IWH Forecast Reports 12(2), 34-90}}
\cvitem{2024}{Germany stuck in stagnation -- private consumption still below pre-pandemic level, \textit{IWH Forecast Reports 12(1), 1-32}}
\cvitem{2023}{Green transformation and the debt brake: implications of additional investment for public finances and private consumption, \textit{IWH Forecast Reports 11(4), 141-159}}
%\cvitem{2023}{Exports and private consumption weak: Germany is waiting for an upturn, \textit{IWH Forecast Reports 11(4), 100-140}}
%\cvitem{2023}{Germany continues its downturn, \textit{IWH Forecast Reports 11(3), 68-97}}
%\cvitem{2023}{Revival in service sectors, but industrial activity remains weak for the time being, \textit{IWH Forecast Reports 11(2), 36-65}}
\cvitem{2023}{Gas storage facilities full economic outlook less gloomy, \textit{IWH Forecast Reports 11(1), 1-34}}
\cvitem{2022}{Economic growth, public finances and greenhouse gas emissions in the medium term, \textit{IWH Forecast Reports 10(4), 146-151}.}
%\cvitem{2022}{No deep recession despite energy crisis and rise in interest rates, \textit{IWH Forecast Reports 10(4), 101-145}.}
%\cvitem{2022}{Energy crisis in Germany, \textit{IWH Forecast Reports 10(3), 68-97}.}
%\cvitem{2022}{War drives up energy prices: High inflation weighs on economy, \textit{IWH Forecast Reports 10(2), 36-65}.}
\cvitem{2022}{Price shock jeopardizes recovery of the German economy, \textit{IWH Forecast Reports 10(1), 2-32}.}

% ---- Teaching ---- %
\section{Teaching Experience}
\cventry{Winter term 2024/2025}{Guest Lecturer}{University of Leipzig (DE)}{}{Module on E-DSGE models} {\ Environmental Macroeconomics (graduate)}
\cventry{Summer term 2022, 2023, 2024}{Teaching Fellow}{University of Trento (IT)}{}{Macroeconomics (undergraduate)}{\ with Prof. Roberto Tamborini}
\cventry{Winter term 2018/2019}{Academic Tutor}{University of Trento (IT)}{}{Introduction to Economics (undergraduate)}{}

% ---- Conferences & Seminars ---- %
\section{Selected Conference Presentations \& Invited Seminars}
\cvitem{2025}{SIE 66th Annual Conference (University of Naples, IT) (forthcoming); VfS Annual Meeting (Köln, DE), EEA Congress (Bordeaux, FR), EAERE Annual Conference (Bergen, NO), RCEA ICEEF2025 (New Jersey City University, US), AERE Annual Conference (Santa Ana Pueblo, US); IAERE Annual Conference (Rome, IT); 2nd International Conference on Climate-Macro-Finance (Bayes Business School, UK)}
\cvitem{2024}{10th Climate Annual Conference (European University Institute, IT); EAERE Annual Conference (Leuven, BE); AERE Annual Conference (Washington DC, US); PhD Seminar (Georgia State University, US); Brown Bag Seminar (Leipzig University, DE)}
\cvitem{2023}{17th International Conference on Computational and Financial Econometrics (HTW Berlin, DE); 24th IWH-CIREQ-GW-BOKERI Macroeconometric Workshop (IWH, DE); SIE 64th Annual Conference (Gran Sasso Science Institute, IT); Conference on Economics of Climate Change and Environmental Policy (University of Orléans, FR); Workshop in Empirical and Theoretical Macroeconomics (King's College London, UK); Doctoral CGDE Workshop (Friedrich Schiller University Jena, DE)}
\cvitem{2022}{AERE Annual Conference (Miami, US); 23rd IWH-CIREQ-GW Macroeconometric Workshop (IWH, DE)}
\cvitem{2021}{22nd IWH-CIREQ-GW Macroeconometric Workshop (IWH, DE); IWH Doctoral Research Seminar (IWH, DE)}
\cvitem{2020}{SIE 61st Annual Conference (Online)}

\begin{comment}
\subsection{Discussions}
\cvitem{2025}{Comments on: "Capital Adjustment Costs and Stranded Assets in an Optimal Clean Energy Transition" by AM. Goeth et al., EAERE, Annual Conference - Bergen (NO)}
\cvitem{2024}{Comments on: "The Horizontal Distributional Impacts of Carbon Pricing: A Behavioral Microsimulation Study for Belgium" by A. De Bevere and G. Grandjean, EAERE, Annual Conference - Leuven (BE)}
\cvitem{2023}{Comments on: "Oil Efficiency, Demand, and Prices: A Tale of Ups and Downs" by M. Bodenstein and L. Guerrieri, 24th IWH-CIREQ-GW-BOKERI Macroeconometric Workshop - IWH (DE); "Hamilton filter with enlarged data set for output gap estimation" by G. Porcellotti, SIE - 64th Annual Conference (IT); "Macroeconomic effects of carbon transition policies: an assessment based on the ECB’s new area- wide model" by G. Coenen et al., Conference on Economics of climate change and environmental policy - University of Orléans (FR)}
\break
% ---- Courses & Summer Schools ---- %
\section{Extra Courses \& Summer Schools}
\cvitem{2024}{Goethe Macro Training School on Heterogeneous-Agent Macroeconomics (Goethe University)}
\cvitem{2023}{Dynamic Resource Economics (Prof. Sjak Smulders); SMEIE - Summer Module on Economics and Institutions in Europe (University of Trento)}
\cvitem{2022}{Macroeconomics of Climate Change (Prof. John Hassler); PhD Summer School on Economic Foundations for Energy and Climate Policies (European University Institute)}
\cvitem{2021}{Bayesian Macroeconometrics (Prof. Joshua Chan); Summer School on Macroeconomics and the science and Art of DSGE modelling (University of Surray)}
\cvitem{2019}{Axel Leijonhufvud Trento Summer School on Macroeconomic topics: Inefficiencies and Coordination failures (University of Trento)}
\end{comment}


% ---- Skills ---- %
\section{Skills}
\cvitemwithcomment{Programming}{R, MATLAB, Dynare, LaTeX, Git; familiar with SQL, Stata}{}
\cvitemwithcomment{Languages}{English (full professional), German (elementary), Italian (native)}{}

% ---- Other Activities ---- %
\section{Other Activities}
\cvitem{2022 -- Present}{Co-organizer of the \textcolor{Maroon}{\href{https://www.iwh-halle.de/en/about-the-iwh/events/detail/26th-iwh-cireq-gw-bokeri-artificial-intelligence-and-macroeconometrics}{IWH-CIREQ-GW-BOKERI Macroeconometric Workshop}}}
\cvitem{2021 -- 2023}{Elected PhD Student Representative, IWH}
\cvitem{2016 -- 2018}{Deputy Coordinator of University Students’ Union, University of Trento}
\cvitem{2014 -- 2018}{Elected Student Representative, University of Trento}

\vfill
% ---- References ---- %

\subsection{References}
\begin{tabular}{@{} p{0.33\textwidth} @{\hspace{0.03\textwidth}} p{0.33\textwidth} @{\hspace{0.03\textwidth}} p{0.33\textwidth} @{}}
  % prima riga
  \begin{minipage}[t]{\linewidth}
    \cvitem{Oliver Holtem\"oller}{%
      \\Halle Institute for Economic Research \& ML University Halle
      \newline
      \textcolor{Maroon}{\href{mailto:oliver.holtemoeller@iwh-halle.de}{%
        oliver.holtemoeller@iwh-halle.de}}
    }
  \end{minipage}
  &
  \begin{minipage}[t]{\linewidth}
    \cvitem{Garth Heutel}{%
      \\Georgia State University -- Andrew Young School of Policy Studies
      \newline
      \textcolor{Maroon}{\href{mailto:gheutel@gsu.edu}{%
        gheutel@gsu.edu}}
    }
  \end{minipage}
  &
  \begin{minipage}[t]{\linewidth}
    \cvitem{Roberto Tamborini}{%
      \\University of Trento -- Department of Economics and \mbox{Management}
      \newline
      \textcolor{Maroon}{\href{mailto:roberto.tamborini@unitn.it}{%
        roberto.tamborini@unitn.it}}
    }
  \end{minipage}
  \\[3em] % spazio tra prima e seconda riga
  % seconda riga (solo prima colonna)
  \begin{minipage}[t]{\linewidth}
    \cvitem{Roland Winkler}{%
      \\FS University Jena -- Faculty of Economics and Business Administration
      \newline
      \textcolor{Maroon}{\href{mailto:roland.winkler@uni-jena.de}{%
        roland.winkler@uni-jena.de}}
    }
  \end{minipage}
  
  & % col 2 vuota
  & % col 3 vuota
\end{tabular}
  
\begin{flushright}
\small Last update: \today
\end{flushright}

\end{document}
